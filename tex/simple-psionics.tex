\documentclass[twocolumn]{dndbook}

% Ensure the graphics path is set correctly
\graphicspath{{../images/}}

\usepackage{graphicx}
\usepackage{geometry}

\begin{document}

\newgeometry{margin=0pt}
\begin{titlepage}
  \noindent
  \includegraphics[width=\paperwidth,height=\paperheight]{531ca660-e751-42c9-9b76-6d50020a53b8.png}
\end{titlepage}
\restoregeometry

% Reset page counter so the next page is page 1
\setcounter{page}{1}

\chapter*{Simple Psionics}

This is a psionics system designed to be very easy to play, without any accounting of psi points or complex powers.
I accomplish this through a series of feats that give existing spells as reskinned abilities.
Unlike usual spells, these psionic powers do not require verbal or somatic components,
but they do have a risk of causing exhaustion on overuse.\par

The inspiration comes from multiple sources:
Eleven from Stranger Things, who has a few simple but powerful abilities,
and the Stephen King novel \emph{Carrie}, and others.
In essence - I do not want to track points, I do not want to create psi-swords.
I just want to play a PC or create an NPC with telepathy, telekinesis, or pyrokinesis in the traditional sense.\par

I did not want this to be the defining feature of a character, so instead of making a class,
I made it a series of feats. Players can take as many of these feats as they want.
I also wanted it to be compatible with various campaign settings.\par

The philosophy here is that psionics is just another form of magic.
The spells used here are all from the D\&D 5e spell list.
The difference is that the psionic powers do not require verbal or somatic components,
and they have a risk of causing exhaustion on overuse.
This simplicity allows easy integration into existing campaigns, without creating new mechanics, and with little or no ambiguity.\par

How psionics fits within the world is up to the DM.
If this is a rare and not-well-known phenomenon in the campaign setting,
the origin feats described below can be overpowered.
In particular, telepathy is powerful, and probably not the best party composition for usual adventures.
It might be a better fit for an NPC, or maybe an higher-level adventure.
Telekinesis can also be very powerful at the hands of creative adventurers.
To balance it, I removed and restricted some of the ability increases, but the DM can remove it altogether.
This can be thought of as the toll that psychic powers take on the user.\par

\section*{Telekinesis}

Telekinesis is a collection of five feats, and players can take as many as they want.
Choose a spellcasting ability between Charisma, Intelligence, or Wisdom, but discuss with the DM.
\footnote{This gives the player or character to flesh out their background, based on the setting and how they acquired their psionic abilities.}
Use it for all of the spell effects granted by that feat.
Constitution saving throw DCs are rolled after each use.
If there is a duration, roll the saving throw at the end of the duration.
If it requires concentration, roll the saving throw when concentration ends.

\subsection{Telekinesis I}
{\emph{Origin Feat}}\par
Whether by mutation, training, or some other means, you have learned to move small objects with your mind.
You gain the following abilities:
\paragraph*{Mage Hand Cantrip} You gain the \emph{Mage Hand} cantrip,
except the hand is invisible and does not require verbal or somatic components.
It is basically undetectable by mundane means,
although \emph{Detect Magic} reveals it.
\paragraph*{Tenser's Floating Desk} You can cast \emph{Tenser's Floating Disk},
without requiring verbal or somatic components.
The disk is invisible and cannot be detected by mundane means,
although \emph{Detect Magic} or similar magic reveals it.
If detected through \emph{True Seeing}, it does not appear as a normal hand, but rather a mass of telekinetic energy.
However, after each use, you must roll a Constitution saving throw (DC 11) or suffer one level of exhaustion.
\paragraph*{Telekinetic Shove} As a bonus action, you can attempt to shove a medium-sized or smaller creature within 15 feet of you
using telekinesis.
The target must succeed on a Strength saving throw (DC 8 plus the spellcasting ability modifier) or be pushed 5 feet away from you.
Optionally, you can choose to pull the target 5 feet closer to you instead.
\paragraph*{Feather Fall} You can cast \emph{Feather Fall} on yourself or a creature you can see within 60 feet of you at will,
without requiring verbal or somatic components. However, after each use, you must roll a Constitution saving throw (DC 11) or suffer one level of exhaustion.

\subsection{Telekinesis II}
{\emph{General Feat (Prerequisite: Telekinesis I)}}\par
\paragraph*{Levitate} You can cast \emph{Levitate} on yourself or one creature within 60 feet of you at will,
without requiring verbal or somatic components. However, after each use, you must roll a Constitution saving throw (DC 12) or suffer one level of exhaustion.
Note that you can lift creatures and stop the levitation at will.
The creature takes damage as per rules if it is dropped from a height, and is prone after the fall.
\paragraph*{Telekinetic Strike} As an action, you can hurl a small or tiny object within 30 feet of you
using telekinesis. Make a ranged spell attack (spellcasting ability modifier + proficiency bonus to hit).
On a hit, the target takes $ 1d6 $ bludgeoning damage, or $ 1d4 $ if the object is tiny.
You can get the object back to you as a bonus action.
\paragraph*{Telekinetic Shove} As per above, but the size is increased to large.
The maximum weight that you can pull or push this way is 500 pounds.
\paragraph*{Gust of Wind} You can cast \emph{Gust of Wind} at will,
without requiring verbal or somatic components. However, after each use, you must roll a Constitution saving throw (DC 12) or suffer one level of exhaustion.
This effect is slightly different than the spell version:
The effect is not a wind but telekinetic force acting on objects and creatures.
As such, it does not disperse gas or vapor effects, nor does it extinguish candles, torches, or similar unprotected flames.

\subsection{Telekinesis III}
{\emph{General Feat (Prerequisite: Telekinesis II)}}\par
\paragraph*{Ability Score Increase} Increase either your Constitution or your spellcasting ability score by 1, to a maximum of 20.
\paragraph*{Fly} You can cast \emph{Fly} on yourself or one creature that you can touch at will,
without requiring verbal or somatic components. However, after each use, you must roll a Constitution saving throw (DC 13) or suffer one level of exhaustion.
\paragraph*{Telekinetic Shove} As per above, but the size is increased to huge.
There is no weight limit for this ability.

\subsection{Telekinesis IV}
{\emph{General Feat (Prerequisite: Telekinesis III)}}\par
\paragraph*{Bigby's Hand} You can cast \emph{Bigby's Hand} at will,
without requiring verbal or somatic components. However, after each use, you must roll a Constitution saving throw (DC 15) or suffer one level of exhaustion.
The hand is invisible, and cannot be detected by mundane means,
although \emph{Detect Magic} and similar magical means reveals it.
If detected through \emph{True Seeing}, it does not appear as a normal hand, but rather a mass of telekinetic energy.
\paragraph*{Blade Barrier} You can cast \emph{Blade Barrier} at will,
without requiring verbal or somatic components. However, after each use, you must roll a Constitution saving throw (DC 16) or suffer one level of exhaustion.
The blades are invisible, and cannot be detected by mundane means,
although \emph{Detect Magic} and similar magical means reveals them.
If detected through \emph{True Seeing}, they do not appear as normal blades, but rather slivers and blades of telekinetic energy.
They still provide cover as per the spell, since they are moving rapidly enough to provide that effect.
\paragraph*{Disintegrate} You can cast \emph{Disintegrate} at will,
without requiring verbal or somatic components. However, after each use, you must roll a Constitution saving throw (DC 16) or suffer one level of exhaustion.
\paragraph*{Telekinesis} You can cast \emph{Telekinesis} at will,
without requiring verbal or somatic components. However, after each use, you must roll a Constitution saving throw (DC 15) or suffer one level of exhaustion.
\paragraph*{Wall of Force} You can cast \emph{Wall of Force} at will,
without requiring verbal or somatic components. However, after each use, you must roll a Constitution saving throw (DC 15) or suffer one level of exhaustion.

\subsection{Telekinesis V}
{\emph{General Feat (Prerequisite: Telekinesis IV)}}\par
\paragraph*{Ability Score Increase} Increase either your Constitution or your spellcasting ability score by 1, to a maximum of 20.
\paragraph*{Telekinetic Shove} As per above, but the size is increased to gargantuan.
\paragraph*{Mordenkainen's Sword} You can cast \emph{Mordenkainen's Sword} at will,
without requiring verbal or somatic components. However, after each use, you must roll a Constitution saving throw (DC 17) or suffer one level of exhaustion.
The sword is invisible, and cannot be detected by mundane means,
although \emph{Detect Magic} and similar magical means reveals it.
If detected through \emph{True Seeing}, it does not appear as a normal sword, but rather a mass of telekinetic energy.

\subsection{Boon of Telekinesis}
{\emph{Epic Boon Feat (Prerequisite: Telekinesis V)}}\par
You have achieved a pinnacle of telekinetic ability.
\paragraph*{Ability Score Increase} Increase either your Constitution or your spellcasting ability score by 1, to a maximum of 30.
\paragraph*{Telekinetic Mastery}
You can now cast any one of the spells in the Prerequisite feats above
a number of times equal to your spellcasting ability modifier + your Constitution modifier (minimum of once),
before you must roll the Constitution saving throw for exhaustion.



\section*{Telepathy}


Your spellcasting ability for Telepathy depends on the effect.

Constitution saving throw DCs are rolled after each use.
If there is a duration, roll the saving throw at the end of the duration.
If it requires concentration, roll the saving throw when concentration ends.

\subsection{Telepathy I}
{\emph{Origin Feat}}\par
Whether by mutation, training, or some other means, you have learned to communicate mentally with others.
You gain the following abilities:
\paragraph*{Friends} You gain the \emph{Friends} cantrip,
except it does not require verbal or somatic components.
Your spellcasting ability for this cantrip is Charisma.
\paragraph*{Charm Person} You can cast \emph{Charm Person} at will,
without requiring verbal or somatic components. However, after each use, you must roll a Constitution saving throw (DC 11) or suffer one level of exhaustion.
\paragraph*{Command} You can cast \emph{Command} at will,
without requiring verbal or somatic components. However, after each use, you must roll a Constitution saving throw (DC 11) or suffer one level of exhaustion.
\paragraph*{Zone of Truth} You can cast \emph{Zone of Truth} at will,
without requiring verbal or somatic components. However, after each use, you must roll a Constitution saving throw (DC 11) or suffer one level of exhaustion.

\subsection{Telepathy II}
{\emph{General Feat (Prerequisite: Telepathy I)}}\par
\paragraph*{Detect Thoughts} You can cast \emph{Detect Thoughts} at will,
without requiring verbal or somatic components. However, after each use, you must roll a Constitution saving throw (DC 12) or suffer one level of exhaustion.
\paragraph*{Hold Person} You can cast \emph{Hold Person} at will,
without requiring verbal or somatic components. However, after each use, you must roll a Constitution saving throw (DC 12) or suffer one level of exhaustion.
\paragraph*{Suggestion} You can cast \emph{Suggestion} at will,
without requiring verbal or somatic components. However, after each use, you must roll a Constitution saving throw (DC 12) or suffer one level of exhaustion.

% TODO: Add Calm Emotions.

\subsection{Telepathy III}
{\emph{General Feat (Prerequisite: Telepathy II)}}\par
\paragraph*{Sending} You can cast \emph{Sending} at will,
without requiring verbal or somatic components. However, after each use, you must roll a Constitution saving throw (DC 13) or suffer one level of exhaustion.
\paragraph*{Compulsion} You can cast \emph{Compulsion} at will,
without requiring verbal or somatic components. However, after each use, you must roll a Constitution saving throw (DC 14) or suffer one level of exhaustion.
\paragraph*{Charm Monster} You can cast \emph{Charm Monster} at will,
without requiring verbal or somatic components. However, after each use, you must roll a Constitution saving throw (DC 14) or suffer one level of exhaustion.
\paragraph*{Locate Creature} You can cast \emph{Locate Creature} at will,
without requiring verbal or somatic components. However, after each use, you must roll a Constitution saving throw (DC 14) or suffer one level of exhaustion.

\subsection{Telepathy IV}
{\emph{General Feat (Prerequisite: Telepathy III)}}\par
\paragraph*{Mass Suggestion} You can cast \emph{Mass Suggestion} at will,
without requiring verbal or somatic components. However, after each use, you must roll a Constitution saving throw (DC 16) or suffer one level of exhaustion.
\paragraph*{Dominate Person} You can cast \emph{Dominate Person} at will,
without requiring verbal or somatic components. However, after each use, you must roll a Constitution saving throw (DC 15) or suffer one level of exhaustion.
\paragraph*{Dream} You can cast \emph{Dream} at will,
without requiring verbal or somatic components. However, after each use, you must roll a Constitution saving throw (DC 15) or suffer one level of exhaustion.
\paragraph*{Rary's Telepathic Bond} You can cast \emph{Rary's Telepathic Bond} at will,
without requiring verbal or somatic components. However, after each use, you must roll a Constitution saving throw (DC 15) or suffer one level of exhaustion.

\subsection{Telepathy V}
{\emph{General Feat (Prerequisite: Telepathy IV)}}\par
\paragraph*{Dominate Monster} You can cast \emph{Dominate Monster} at will,
without requiring verbal or somatic components. However, after each use, you must roll a Constitution saving throw (DC 18) or suffer one level of exhaustion.
\paragraph*{Glibness} You can cast \emph{Glibness} at will,
without requiring verbal or somatic components. However, after each use, you must roll a Constitution saving throw (DC 18) or suffer one level of exhaustion.
\paragraph*{Power Word Stun} You can cast \emph{Power Word Stun} at will,
without requiring verbal or somatic components. However, after each use, you must roll a Constitution saving throw (DC 18) or suffer one level of exhaustion.
\paragraph*{Telepathy} You can cast \emph{Telepathy} at will,
without requiring verbal or somatic components. However, after each use, you must roll a Constitution saving throw (DC 18) or suffer one level of exhaustion.
% TODO: Add Mind Blank


\subsection{Boon of Telepathy}
{\emph{Epic Boon Feat (Prerequisite: Telepathy V)}}\par
You have achieved a pinnacle of telepathic ability.
\paragraph*{Telepathic Communication}
You can communicate telepathically with any creature you can see within 1 mile of you.
You can also cast \emph{Rary's Telepathic Bond}, \emph{Telepathy}, \emph{Sending} at will,
without requiring verbal or somatic components. This does not require a Constitution saving throw.
\paragraph*{Power Word Kill} You can cast \emph{Power Word Kill} at will,
without requiring verbal or somatic components. However, after each use, you must roll a Constitution saving throw (DC 19) or suffer one level of exhaustion.

\section{Pyrokinesis}

Pyrokinesis is a collection of five feats, and players can take as many as they want.
Choose a spellcasting ability between Charisma, Intelligence, or Wisdom, but discuss with the DM.
\footnote{This gives the player or character to flesh out their background, based on the setting and how they acquired their psionic abilities.}
Constitution saving throw DCs are rolled after each use.
If there is a duration, roll the saving throw at the end of the duration.
If it requires concentration, roll the saving throw when concentration ends.\par

I focused on spells that start or control a fire, and omitted the projectile spells.

\subsection{Pyrokinesis I}
{\emph{Origin Feat}}\par
Whether by mutation, training, or some other means, you are able to start and control fires at will.
\paragraph*{Produce Flame Cantrip} You can cast \emph{Produce Flame} at will,
without requiring verbal or somatic components. Cantrip upgrade applies based on your level, and upgrades as you level up.
\paragraph*{Elementalism Cantrip} You can cast the cantrip \emph{Elementalism}.
However, you can only create fire effects with it - so you can only ``beckon fire'' and ``sculpt'' only fire.
You can cast it at will, without requiring verbal or somatic components.
\paragraph*{Burning Hands} You can cast \emph{Burning Hands} at will,
without requiring verbal or somatic components. However, after each use, you must roll a Constitution saving throw (DC 11) or suffer one level of exhaustion.

\subsection{Pyrokinesis II}
{\emph{General Feat (Prerequisite: Pyrokinesis I)}}\par
\paragraph*{Heat Metal} You can cast \emph{Heat Metal} at will,
without requiring verbal or somatic components. However, after each use, you must roll a Constitution saving throw (DC 12) or suffer one level of exhaustion.
\paragraph*{Flaming Sphere} You can cast \emph{Flaming Sphere} at will,
without requiring verbal or somatic components. However, after each use, you must roll a Constitution saving throw (DC 12) or suffer one level of exhaustion.
\paragraph*{Burning Hands} You can now cast \emph{Burning Hands} as a 2nd-level spell at will,
which increases the damage to $ 4d6 $. However, if you do so, the Constitution saving throw is DC 12.
You can still cast it as a 1st-level spell as well, for a lower DC on the Constitution saving throw.

\subsection{Pyrokinesis III}
{\emph{General Feat (Prerequisite: Pyrokinesis II)}}\par
\paragraph*{Ability Score Increase} Increase either your Constitution or your decided spellcasting ability by 1, to a maximum of 20.
\paragraph*{Wall of Fire} You can cast \emph{Wall of Fire} at will,
without requiring verbal or somatic components. However, after each use, you must roll a Constitution saving throw (DC 14) or suffer one level of exhaustion.
\paragraph*{Burning Hands} You can now cast \emph{Burning Hands} as a 3rd or 4th-level spell at will,
which increases the damage to $ 4d6 $ or $ 5d6 $. However, if you do so, the Constitution saving throw is DC 13 or 14 respectively.
You can still cast it as a 1st- or 2nd-level spell as well, for a lower DC on the Constitution saving throw.
\paragraph*{Flaming Sphere} You can now cast \emph{Flaming Sphere} as a 3rd- or 4th-level spell at will.
However, if you do so, the Constitution saving throw is DC 13 or 14 respectively.
You can still cast it as a 2nd-level spell as well, for a lower DC on the Constitution saving throw.
\paragraph*{Fire Shield} You can cast \emph{Fire Shield} at will,
without requiring verbal or somatic components. However, after each use, you must roll a Constitution saving throw (DC 14) or suffer one level of exhaustion.

\subsection{Pyrokinesis IV}
{\emph{General Feat (Prerequisite: Pyrokinesis III)}}\par
\paragraph*{Burning Hands} You can now cast \emph{Burning Hands} as a 5th- or 6th-level spell at will,
which increases the damage to $ 6d6 $ or $ 7d6 $. However, if you do so, the Constitution saving throw is DC 15 or 16 respectively.
You can still cast it as a 1st-, 2nd-, 3rd-, or 4th-level spell as well, for a lower DC on the Constitution saving throw.
\paragraph*{Flaming Sphere} You can now cast \emph{Flaming Sphere} as a 5th- or 6th-level spell at will.
However, if you do so, the Constitution saving throw is DC 15 or 16 respectively.
You can still cast it as a 2nd-, 3rd-, or 4th-level spell as well, for a lower DC on the Constitution saving throw.
\paragraph*{Wall of Fire} You can now cast \emph{Wall of Fire} as a 6th- or 7th-level spell at will.
However, if you do so, the Constitution saving throw is DC 15 or 16 respectively.
You can still cast it as a 4th- or 5th-level spell as well, for a lower DC on the Constitution saving throw.
\paragraph*{Resistance to Fire} You have resistance to fire damage.

\subsection{Pyrokinesis V}
{\emph{General Feat (Prerequisite: Pyrokinesis IV)}}\par
\paragraph*{Ability Score Increase} Increase either your Constitution or your decided spellcasting ability by 1, to a maximum of 20.
\paragraph*{Fire Storm} You can cast \emph{Fire Storm} at will,
without requiring verbal or somatic components. However, after each use, you must roll a Constitution saving throw (DC 17) or suffer one level of exhaustion.
\paragraph*{Burning Hands} You can now cast \emph{Burning Hands} as a 7th- or 8th-level spell at will,
which increases the damage to $ 8d6 $ or $ 9d6 $. However, if you do so, the Constitution saving throw is DC 17 or 18 respectively.
You can still cast it as a lower-level spell as well, for a lower DC on the Constitution saving throw.
\paragraph*{Flaming Sphere} You can now cast \emph{Flaming Sphere} as a 7th- or 8th-level spell at will.
However, if you do so, the Constitution saving throw is DC 17 or 18 respectively.
You can still cast it as a lower-level spell as well, for a lower DC on the Constitution saving throw.
\paragraph*{Wall of Fire} You can now cast \emph{Wall of Fire} as a 7th- or 8th-level spell at will.
However, if you do so, the Constitution saving throw is DC 17 or 18 respectively.
You can still cast it as a lower-level spell as well, for a lower DC on the Constitution saving throw.




\section{Cryokinesis}

If Pyrokinesis is a possibility in a D\&D setting, then Cryokinesis should be as well.
Cryokinesis is a collection of five feats, and players can take as many as they want.
Choose a spellcasting ability between Charisma, Intelligence, or Wisdom, but discuss with the DM.
\footnote{This gives the player or character to flesh out their background, based on the setting and how they acquired their psionic abilities.}
Constitution saving throw DCs are rolled after each use.
If there is a duration, roll the saving throw at the end of the duration.
If it requires concentration, roll the saving throw when concentration ends.\par

In designing this feat path, I focused on spells that create or control ice and cold.
I included \emph{Ray of Frost}, which is a staple cantrip for cold-themed casters.
I also included \emph{Ice Knife}, which is a fun spell that creates ice projectiles,
but created an additional requirement of having access to water in your proximity to be able to cast it.


\subsection{Cryokinesis I}
{\emph{Origin Feat}}\par
Whether by mutation, training, or some other means, you are able to create and control ice and cold at will.
\paragraph*{Frostbite Cantrip} You can cast \emph{Frostbite} at will,
without requiring verbal or somatic components. Cantrip upgrade applies based on your level, and upgrades as you level up.
\paragraph*{Ray of Frost} You can cast \emph{Ray of Frost} at will,
without requiring verbal or somatic components. Cantrip upgrade applies based on your level, and upgrades as you level up.
\paragraph*{Ice Knife} You can cast \emph{Ice Knife} at will,
without requiring verbal or somatic components. However, after each use, you must roll a Constitution saving throw (DC 11) or suffer one level of exhaustion.
In addition, to be able to cast this spell, you must have access to water (or ice, or steam) in your proximity - at least 1 pint within 30 feet of you.

\subsection{Cryokinesis II}
{\emph{General Feat (Prerequisite: Cryokinesis I)}}\par
\paragraph*{Frostbite Cantrip} Your \emph{Frostbite} cantrip deals $ 2d6 $ damage instead of $ 1d6 $ damage.
\paragraph*{Ray of Frost Cantrip} Your \emph{Ray of Frost} cantrip deals $ 2d8 $ damage instead of $ 1d8 $ damage.
\paragraph*{Ice Knife} You can now cast \emph{Ice Knife} as a 2nd-level spell at will,
which increases the damage to $ 3d6 $ on a hit. However, if you do so, the Constitution saving throw is DC 12.
You can still cast it as a 1st-level spell as well, for a lower DC on the Constitution saving throw.


\subsection{Cryokinesis III}
{\emph{General Feat (Prerequisite: Cryokinesis II)}}\par
\paragraph*{Ability Score Increase} Increase either your Constitution or your Charisma by 1, to a maximum of 20.
\paragraph*{Ice Storm} You can cast \emph{Ice Storm} at will,
without requiring verbal or somatic components. However, after each use, you must roll a Constitution saving throw (DC 14) or suffer one level of exhaustion.
\paragraph*{Ice Knife} You can now cast \emph{Ice Knife} as a 3rd- or 4th-level spell at will,
which increases the damage to $ 4d6 $ or $ 5d6 $ on a hit. However, if you do so, the Constitution saving throw is DC 13 or 14 respectively.
You can still cast it as a 1st- or 2nd-level spell as well, for a lower DC on the Constitution saving throw.

\subsection{Cryokinesis IV}
{\emph{General Feat (Prerequisite: Cryokinesis III)}}\par
\paragraph*{Frostbite Cantrip} Your \emph{Frostbite} cantrip deals $ 3d6 $ damage.
\paragraph*{Ray of Frost Cantrip} Your \emph{Ray of Frost} cantrip deals $ 3d8 $ cold damage.
\paragraph*{Cone of Cold} You can cast \emph{Cone of Cold} at will,
without requiring verbal or somatic components. However, after each use, you must roll a Constitution saving throw (DC 15) or suffer one level of exhaustion.
\paragraph*{Otiluke's Freezing Sphere} You can cast \emph{Otiluke's Freezing Sphere} at will,
without requiring verbal or somatic components. However, after each use, you must roll a Constitution saving throw (DC 16) or suffer one level of exhaustion.
\paragraph*{Ice Knife} You can now cast \emph{Ice Knife} as a 5th- or 6th-level spell at will,
which increases the damage to $ 6d6 $ or $ 7d6 $ on a hit. However, if you do so, the Constitution saving throw is DC 15 or 16 respectively.
You can still cast it as a lower-level spell as well, for a lower DC on the Constitution saving throw.
\paragraph*{Wall of Ice} You can cast \emph{Wall of Ice} at will,
without requiring verbal or somatic components. However, after each use, you must roll a Constitution saving throw (DC 16) or suffer one level of exhaustion.
\paragraph*{Resistance to Cold} You have resistance to cold damage.

\subsection{Cryokinesis V}
{\emph{General Feat (Prerequisite: Cryokinesis IV)}}\par
\paragraph*{Ability Score Increase} Increase either your Constitution or your Charisma by 1, to a maximum of 20.
\paragraph*{Frostbite Cantrip} Your \emph{Frostbite} cantrip deals $ 4d6 $ damage.
\paragraph*{Ray of Frost Cantrip} Your \emph{Ray of Frost} cantrip deals $ 4d8 $ cold damage.
\paragraph*{Ice Knife} You can now cast \emph{Ice Knife} as a 7th- or 8th-level spell at will,
which increases the damage to $ 8d6 $ or $ 9d6 $ on a hit. However, if you do so, the Constitution saving throw is DC 17 or 18 respectively.
You can still cast it as a lower-level spell as well, for a lower DC on the Constitution saving throw.
\paragraph*{Otiluke's Freezing Sphere} You can now cast \emph{Otiluke's Freezing Sphere} as a 7th- or 8th-level spell at will.
However, if you do so, the Constitution saving throw is DC 17 or 18 respectively.
You can still cast it as a lower-level spell as well, for a lower DC on the Constitution saving throw.
\paragraph*{Wall of Ice} You can now cast \emph{Wall of Ice} as a 7th- or 8th-level spell at will.
However, if you do so, the Constitution saving throw is DC 17 or 18 respectively.
You can still cast it as a lower-level spell as well, for a lower DC on the Constitution saving throw.


\end{document}
